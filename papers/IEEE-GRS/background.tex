\section{Background and Previous Work} \label{sec:background_perious_work}

In this section, the previous work done in this application is reviewed. The need for dune crest-line detection has risen recently due to many research applications in this field. Therefore, understanding the concepts and properties of dune fields is important. Many studies have been done on the topic of dune fields, with the ultimate goal of understanding how dunes form, behave, and which factors contribute to the systems. In \cite{Kocurek_Ewing}, an explanation of how dune-field patterns emerge is proposed and the degree of complexity from the standpoint of self-organizing system is discussed. Dune patterns are classified into two basic categories: simple or complex. According to this paper, a simple pattern is defined as having a single pattern type. Complex patterns are said to potentially have multiple spatially superimposed pattern types. Simple patterns trend towards a better ordering, so long as the wind regime remains constant. Changes in the trend of wind pattern will cause reorientation. Complex patterns can then be interpreted as the superposition of many generations of wind regimes.

Complexity can be measured using pattern analysis, measuring parameters such as the crest length, orientation, spacing,	defect density, and other properties. These properties would be very useful information to extract for the dune detection. Finding crest-lines	can help us identify the patterns, extracting meaningful data such as junctions, terminations, mergers, linking, and other interesting processes explained in \cite{Kocurek_Ewing}. In later work, \cite{Ewing_Peyret_Kocurek_Bourke} studied the dune field pattern formations on the north polar region of Mars. These dunes are thought to be mostly inactive, with relatively no movement over a period of 4 to 15 Martian years. The reason this region is interesting is because there are two nearly orthogonal crest-line orientations present in the region. This complex system of superposition of multiple patterns in this example showcases a set of \emph{primary} and \emph{secondary} crest-lines which would be valuable data to automatically segment. The \emph{primary}	dunes are the largest-scale dunes, extend over the entire length of the image set, and contain many \emph{Y} junctions, an indication of well-organized linear dunes. In contrast, \emph{secondary} dunes are rounded, have less defined features, and are perpendicular to the main primary crest-lines.

Other interesting features are the so-called \emph{slipfaces}. These typically appear along the \emph{primary} crest-lines, in areas of intersection with the \emph{primary} and \emph{secondary} crest-lines. Another feature are the \emph{Wind Ripples}. The ripples are present on the surface of most dunes with the exception	of \emph{slipfaces}. To detect	these types of features, a higher resolution image is a required, as those types of features are very small compared to the scale of the \emph{primary} and \emph{secondary}	dunes. \emph{Interdune Areas} are features specific to the area studied in \cite{Ewing_Peyret_Kocurek_Bourke},	have a polygonal shape, and are indicative of ice. The authors also describe an in depth statistical analysis of the features to compute the flow fields and understand the geomorphic relationships which are present in the area. If these features can be extracted, a similar statistical analysis can be automatically computed.

In the most recent work, \cite{Multi_spatial_analysis_aeolian_dune_field_patterns} study the aeolian dune-fields at different scales. Scale is an important factor to consider because aeolian dune-fields patterns can vary over a wide range of scales, both spatially and temporally. Being able to measure the change of scale over time is important in order to investigate the environmental conditions of the studied region. Aeolian dunes are developed typically in transitions from sand patches, to proto-dunes, to dunes, to dune-field patterns. Complex dune patterns are usually a juxtaposition of simple dune patterns are multiple scales. To summarize, being able to detect the crest-lines and other types of features at multiple scales is invaluable.

In other related work, \cite{Application_spatial_cross_correlation_detection_submarine_dunes} studied the migration of submarine sand dunes. The dataset used in this paper includes digital terrain models (DTMs) retrieved from high-density multibeam echosounders (MBES) taken of submarine sand dunes along the coast of New Brunswick. In order to measure the migration of these types of dunes, the motion was measured by simply subtracting the DTMs from sequences over time. The implementation uses a simple cross-correlation to find similarities from one set to the next. From the correlation matches found, the migration vector can be computed. The migration data processing can extract the flow fields of the dunes.

\subsection*{Dune Detection}
There are two main methods used to detect dune landforms and crest-lines: appearance-based and machine learning methods. In \cite{2012_automated_extraction_sand_dunes_egypt}, a Geographic Information System (GIS, \cite{gis_article}) based model is proposed and developed to study sand dune migration patterns, with the ultimate goal of predicting threats to roads, irrigation networks, water resources, urban ares, agriculture and infrastructures. Multi-temporal images of a region were collected, enhanced, and a simple image subtraction technique is used to compute the migration shift of dunes.

An automated extraction of dune features approach is proposed in \cite{2015_automated_mapping_of_linear_dunefield}, which uses the Sobel gradient operator to detect crest-lines. In this approach, the authors claim that the Canny edge detector provided poor results, and the Sobel operator generated more consistent results. The gradient magnitude of the Sobel operator is used to threshold out the weaker edges. The use of a histogram to compute the orientations of the gradients was found to have a bimodal distribution, with one of the modes being the dune edges. Dune candidates are determined based on their gradient magnitude which filter out weaker candidates unless they are near strong candidates. This research addresses the issue of determining crest-lines versus valleys or shadows, explained in section \ref{subsec:challenges}, by using the histogram of gradients. Examination of the histogram reveals that they had a strongly bimodal distribution for which one peak represents crest-lines and the other represents the valleys or shadows. The assumption made in their research is that the stronger peak represents the crest-lines, and therefore all gradients which belong to the weaker peak (within 90 degrees of the peak) are filtered out, which may not always be valid and is addressed in this research.

In \cite{2016_comparisons_crest_line_extraction_marine_dunes}, an overview of various line detection algorithms are reviewed in the context of a marine dune crest-line extraction method. The research evaluates four main appearance-based approaches (\cite{2005_topology_driven_algorithms_for_ridge_extraction,2005_smooth_feature_lines_surface_meshes,2004_ridge_valley_lines_meshes_surface_fitting}) to detecting ridges from MultiBeam Echo-sounder System (MBES) data maps. The data for these approaches is essentially 3D depth maps of underwater dune forms. The challenge therefore is to trace the ridges to extract the dune crest-lines. Many of the research techniques to detect and identify various landforms use commonly known machine learning techniques. In \cite{2006_automated_classification_landform_elements}, object-orientated image analysis is used to classify the type of landform elements found in Digital Terrain Models (DTM). Features such as elevation, curvature, and slope gradient are used to construct a classification model, which categorizes landform elements into nine distinct base shapes. The shapes are somewhat abstract and generalized, but the concepts presented in this paper could be applied to crest-line detection.

The research done in \cite{2007_Machine_Learning_tools_automatic_mapping_mars} developed a machine learning based approach to map and categorize various landforms on Mars. In this research, it is shown that regions can be clustered using unsupervised learning methods, and labeled using supervised learning, given a dataset which has been manually labeled. The popular Support Vector Machines classifier is used to predict the label of the clustered regions. The research done in \cite{2013_sar_image_automated_detection_dune_area} uses a simple approach in which the correlation is computed between two images taken at different times within a year. Various window sizes are used to account for dunes of different scale. An interesting observation is made; as the window size grows larger, the correlation values tend to increase in areas where dunes are present, while remaining relatively constant in areas without dunes. A supervised learning method is used to classify areas which dunes are present. In \cite{BandeiraMarques}, a supervised learning approach is used, training classifiers such as Support Vector Machine and Random Forests to detect dune structures on Mars. The method proposed in this paper is to classify small (40 by 40 pixels) cells in a quantized image grid. In each cell, features are computed based on the image gradients, using both phase and magnitude. In order to classify a cell as either a dune or not a dune, the features of both the cell and the cell's neighboring cells are used. The features extracted are then used to train the machine learning method, which is used to then predict if a cell is a dune.

Although this type of method has typically shown very good results, there are a few drawbacks with using supervised learning approaches. These types of methods usually require a fairly large labeled dataset which may not always be available. Anytime a dataset is constructed for this purpose, it is important to provide a large number of examples of different types of dunes, in order to get a robust representation of the problem set. In \cite{BandeiraMarques}, 230 labeled images were used to train and test the method, and have a decent representation of various dune types. Another drawback of this approach is the use of cells, for which fixed-sized cells may not be scale invariant. Also, quantizing an image into larger cells	will affect localization accuracy of the dunes. If the application requires higher localization accuracy, this type of supervised learning approach may not be suitable. In \cite{2011_neural_network_based_dunal_landform_mapping}, texture features are extracted from images to train a multilayer perception or neural network to generate landform maps. The texture features used are the Grey Level Co-occurrence Matrix (GLCM \cite{1973_textural_features_image_classification}), are fed into the neural network with goal of characterizing various landforms. A critical problem to solve is the selection of the window size for computing the texture features. Reportedly, smaller window sizes, relative to the spatial resolution, negatively affect the classification accuracy. A large dataset was used to train the neural network, and shows promising results.

The most current and direct research application to dune crest-line detection is presented in \cite{vaz_object_based_dune_analysis}. An object-based image analysis method is used to extract crest-lines from dune field datasets on mars. The main approach extracts features from the dune field images, and trains an artificial neural network classifier. Once the classifier is trained, the crest-lines can be extracted and the dune field can be analyzed. The object-based image analysis methodology is very similar to our research, and the datasets used in this paper were acquired. Section \ref{subsec:results-and-discussion} shows the comparison of the results reported in \cite{vaz_object_based_dune_analysis} and our results, using our computed dune metrics. The remainder of this paper discusses the various methodologies (section \ref{sec:methodology}) implemented as part of this research. The approaches include both appearance-based and machine learning methods to crest-line detection. The results of each approach are then presented in section \ref{sec:experimental_evaluation}.